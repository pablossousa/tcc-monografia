\chapter{Trabalhos Relacionados}
\label{chap:trabalhos-relacionados}

O uso de tecnologias biométricas em ambientes educacionais tem sido explorado em diferentes contextos, tanto para controle de acesso quanto para monitoramento de estudantes e otimização de serviços institucionais. Assim como discutido na fundamentação teórica, essas soluções buscam reduzir fraudes, automatizar processos e aumentar a confiabilidade dos sistemas, substituindo mecanismos baseados exclusivamente na posse de credenciais físicas ou em procedimentos manuais. Nesta seção, são apresentados e analisados dois trabalhos diretamente relacionados à proposta deste estudo, permitindo situá-la no contexto da literatura e evidenciar suas principais diferenças e contribuições.

Um dos trabalhos mais próximos ao contexto deste estudo é o desenvolvido por Silva Júnior, intitulado \textit{UNIACCESS: uma aplicação que auxilia no gerenciamento do acesso e consumo das refeições do restaurante estudantil do IFPB – Campus Cajazeiras} \cite{silvajunior2025}. O autor propõe uma aplicação integrada composta por três serviços: um sistema \textit{web} para gerenciamento dos alunos autorizados, um serviço biométrico para identificação dos discentes na entrada do restaurante e um aplicativo móvel para confirmação prévia da intenção de consumo das refeições. O objetivo principal é automatizar o controle de acesso ao restaurante estudantil, reduzir filas, minimizar fraudes e contribuir para a diminuição do desperdício de alimentos por meio de um planejamento mais preciso da quantidade de refeições a serem preparadas.

A solução apresentada por Silva Júnior \cite{silvajunior2025} utiliza a biometria por impressão digital como mecanismo de autenticação dos estudantes. Essa abordagem apresenta vantagens em relação a controles manuais, pois associa o acesso a uma característica biométrica do usuário e reduz a possibilidade de uso indevido por terceiros. Entretanto, essa escolha implica a necessidade de sensores específicos, além do armazenamento de um dado biométrico considerado altamente sensível. Além disso, o foco do trabalho está voltado principalmente à integração dos módulos do sistema, sem uma análise aprofundada de representações vetoriais, métricas de similaridade ou do comportamento do sistema sob diferentes limiares de decisão.

O presente trabalho compartilha com a proposta de Silva Júnior \cite{silvajunior2025} o mesmo cenário de aplicação, porém adota uma abordagem técnica distinta. Em vez de utilizar impressão digital, propõe-se o uso de reconhecimento facial baseado em \textit{embeddings}, explorando modelos pré-treinados e métricas de similaridade no espaço vetorial. Essa escolha possibilita o reaproveitamento de equipamentos já existentes, como câmeras comuns, reduzindo a necessidade de \textit{hardware} dedicado. Além disso, o foco desloca-se para a análise do comportamento das representações faciais no espaço métrico, considerando a separação entre identidades e as taxas FAR e FRR em função do limiar de decisão.

Outro trabalho relevante é a dissertação de Rocha, intitulada \textit{Reconhecimento facial de alunos de escola pública no uso de ônibus escolar em cidade inteligente} \cite{rocha2023}. Nesse estudo, o autor propõe uma plataforma para monitoramento de estudantes com o objetivo de mitigar problemas como evasão escolar e desaparecimento de crianças e adolescentes. O sistema utiliza reconhecimento facial para identificar os alunos em diferentes pontos do trajeto escolar e integra aplicações \textit{web} e móveis para acompanhamento por responsáveis e autoridades. Do ponto de vista técnico, são exploradas diferentes técnicas de inteligência artificial, incluindo descritores clássicos e redes neurais convolucionais, com resultados experimentais expressivos em termos de acurácia e demais métricas de classificação.

A proposta de Rocha \cite{rocha2023} evidencia o potencial do reconhecimento facial em contextos educacionais, especialmente no que se refere ao monitoramento automático de presença. Entretanto, o foco principal está na arquitetura do sistema e no desempenho de classificadores, tratando o problema predominantemente como uma tarefa de classificação supervisionada. Não há uma ênfase específica na análise do espaço de \textit{embeddings} nem no uso de métricas de similaridade com limiares ajustáveis para decisões de verificação.

O presente trabalho aproxima-se da proposta de Rocha \cite{rocha2023} ao adotar o reconhecimento facial como tecnologia central, mas diferencia-se ao enquadrar explicitamente o problema no contexto do aprendizado métrico. Utilizam-se \textit{embeddings} faciais e métricas de distância para estudar a separação entre identidades e o compromisso entre segurança e usabilidade por meio das taxas \acs{FAR} e \acs{FRR}. Além disso, enquanto Rocha trabalha com dados reais de estudantes, esta pesquisa opta pelo uso de uma base de dados sintética, de modo a mitigar riscos éticos e legais associados ao tratamento de dados biométricos sensíveis.

De forma geral, os dois trabalhos analisados demonstram a relevância do uso de biometria em ambientes educacionais, seja por meio de impressão digital para controle de acesso a serviços institucionais \cite{silvajunior2025}, seja por meio de reconhecimento facial para monitoramento de estudantes \cite{rocha2023}. A proposta deste trabalho insere-se nesse mesmo panorama, mas contribui ao direcionar o foco para a análise do reconhecimento facial baseado em \textit{embeddings} como alternativa ao uso de cartões \acs{NFC}, enfatizando a simplicidade de integração, o reaproveitamento de infraestrutura existente e a avaliação quantitativa do comportamento do sistema sob diferentes critérios de decisão.

A análise dos trabalhos relacionados permitiu identificar diferentes abordagens para o uso de biometria em contextos educacionais e de controle de acesso, bem como suas principais características e limitações. Com base nessas referências e nos conceitos discutidos anteriormente, o próximo capítulo apresenta a metodologia adotada neste trabalho, descrevendo a construção da base de dados, o \textit{pipeline} de reconhecimento facial e o protocolo experimental utilizado para avaliar a proposta.
