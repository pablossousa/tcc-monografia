\chapter{Cronograma de Atividades}

Este capítulo apresenta o planejamento das atividades a serem desenvolvidas no âmbito do Projeto Final de Curso, definindo os objetivos, as etapas de execução, o cronograma e a forma de acompanhamento do projeto.

\section{Objetivos das Atividades}

Desenvolver e validar um protótipo funcional de sistema de controle de acesso baseado em reconhecimento facial, com armazenamento seguro de \textit{embeddings}, integrando um mecanismo de entrada por meio de teclado virtual associado a uma \textit{tag} \acs{RFID}, e avaliar seu desempenho em cenários reais do \acs{CEFET-MG} com diferentes níveis de fluxo de pessoas.

\section{Descrição das Atividades}

\begin{itemize}
\item[1] \textbf{Apresentação do PFC I:} preparar e apresentar o projeto à banca avaliadora, contemplando a proposta, os objetivos, a metodologia e o planejamento das atividades;
\item[2] \textbf{Construção do protótipo:} desenvolver o protótipo do sistema de reconhecimento facial utilizando bibliotecas e tecnologias do estado da arte, contemplando o armazenamento seguro dos \textit{embeddings} e a integração entre os módulos do sistema;
\item[3] \textbf{Proposta e implementação do dispositivo de entrada:} projetar e implementar um teclado virtual que associe a imagem capturada a uma \textit{tag} de cartão \acs{RFID}, simulando o processo de autenticação no controle de acesso;
\item[4] \textbf{Testes iniciais com dados sintéticos:} realizar experimentos controlados utilizando imagens faciais sintéticas para validação preliminar do funcionamento e da robustez do sistema;
\item[5] \textbf{Testes finais com dados reais em ambiente de baixa ou moderada movimentação:} conduzir testes no contexto da biblioteca, avaliando o desempenho do sistema em um cenário com fluxo reduzido ou moderado de pessoas;
\item[6] \textbf{Testes finais com dados reais em ambiente de alta movimentação:} conduzir testes no contexto do restaurante estudantil, avaliando o comportamento do sistema em um cenário de grande fluxo de usuários;
\item[7] \textbf{Documentação e redação do PFC II:} consolidar a documentação do projeto, descrevendo o desenvolvimento, os experimentos realizados, os resultados obtidos e as análises correspondentes;
\item[8] \textbf{Apresentação do PFC II:} preparar e apresentar o trabalho final à banca avaliadora.
\end{itemize}


\section{Cronograma de Atividades}

\begin{table}[!ht]
 \centering
\caption{Cronograma de atividades.}
\begin{center}
{\footnotesize{\begin{tabular}{c|c|c|c|c|c} \hline
Atividade ($\downarrow$) / Mês ($\rightarrow$) & Fev & Mar & Abr & Mai & Jun \\ \hline
$1$ &  & $\surd$ &  &  &  \\ \hline
$2$ & $\surd$ & $\surd$ &  &  &  \\ \hline
$3$ &  & $\surd$ & $\surd$ &  &  \\ \hline
$4$ &  &  & $\surd$ &  &  \\ \hline
$5$ &  &  &  & $\surd$ &  \\ \hline
$6$ &  &  &  & $\surd$ & $\surd$ \\ \hline
$7$ &  & $\surd$ & $\surd$ & $\surd$ & $\surd$ \\ \hline
$8$ &  &  &  &  & $\surd$ \\ \hline
\end{tabular}}}
\end{center}
\footnotesize Fonte: Elaborado pelo autor, 2026.
\end{table}

\section{Local de Desenvolvimento das Atividades}

As atividades serão desenvolvidas em Divinópolis, Minas Gerais, no ambiente acadêmico do \acs{CEFET-MG} e no ambiente doméstico do aluno, conforme a natureza de cada etapa do projeto.

\section{Metodologia de Acompanhamento}

O acompanhamento do projeto será realizado por meio de reuniões periódicas com o orientador, com o objetivo de discutir o andamento das atividades, avaliar os resultados parciais e ajustar o planejamento quando necessário, garantindo a coerência entre o desenvolvimento técnico do sistema e a redação do trabalho final.
