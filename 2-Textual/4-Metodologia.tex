\chapter{Metodologia}
\label{chap:metodologia}

Este trabalho adota uma metodologia experimental com o objetivo de avaliar a viabilidade tecnológica do uso de reconhecimento facial como alternativa ao controle de acesso baseado em cartões \acs{NFC}, no contexto acadêmico do \acs{CEFET-MG}.

A \ref{fig:diagrama1} apresenta uma visão geral do pipeline metodológico adotado neste trabalho, destacando as principais etapas do processo experimental, desde a aquisição das imagens faciais até a avaliação do reconhecimento para fins de controle de acesso.

\begin{figure}[!htb]
    \centering
    \begin{minipage}{1.0\textwidth}
        \caption{\label{fig:diagrama1} Pipeline metodológico adotado.}
        \includegraphics[width=\textwidth]{Imagens/diagramaBlocos.png}
        \caption*{\footnotesize Fonte: Elaborado pelo autor, 2026.}
    \end{minipage}
\end{figure}

Nesse contexto, busca-se verificar se \textit{embeddings} faciais extraídos por modelos de reconhecimento permitem a separação adequada entre diferentes identidades por meio de métricas de similaridade, possibilitando decisões confiáveis de autenticação e identificação \cite{ge2018hierarchical}. Considerando as restrições éticas e de privacidade associadas ao uso de dados biométricos reais \cite{north2020biometric}, a metodologia emprega uma base de dados sintética e segue um conjunto estruturado de etapas, descritas a seguir, de forma a garantir reprodutibilidade e clareza na avaliação da abordagem proposta.

%---------------------------------------------------------

\section{Construção da base de dados sintética}

A construção da base de dados utilizada neste estudo teve como objetivo viabilizar a realização de experimentos de reconhecimento facial sem a utilização de dados biométricos reais, em conformidade com princípios éticos e com a \acs{LGPD} (Lei nº 13.709/2018) \cite{lgpd2018lei}. Para esse fim, foi empregada uma base de imagens faciais sintéticas geradas artificialmente por meio de redes generativas adversariais (\acs{GANs}), disponibilizadas publicamente pelo projeto \textit{This Person Does Not Exist}, o qual produz imagens fotorrealistas de rostos inexistentes, disponível em \cite{thispersondoesnotexist}.

Foram coletadas 400 imagens faciais sintéticas, cada uma representando uma identidade distinta. As imagens foram armazenadas em formato digital e organizadas em diretórios específicos do projeto, adotando-se um padrão de nomenclatura que permitisse a associação direta entre cada imagem e seu respectivo identificador de identidade. A utilização de dados sintéticos assegura que não exista qualquer vínculo com indivíduos reais, eliminando riscos de identificação pessoal e atendendo às diretrizes de privacidade estabelecidas pela \acs{LGPD} \cite{lgpd2018lei}.

Com o objetivo de estruturar adequadamente os experimentos, a base de dados foi particionada em dois conjuntos principais: um conjunto de desenvolvimento, composto por 300 identidades, e um conjunto de teste final, composto por 100 identidades distintas. O conjunto de desenvolvimento foi subdividido em três subconjuntos, denominados treino, validação e teste, utilizados exclusivamente para organização experimental, construção da base de referência e calibração de limiares, contendo, respectivamente, 240, 30 e 30 identidades. Embora não haja treinamento ou ajuste dos modelos de reconhecimento, essa subdivisão permite a avaliação controlada do comportamento do sistema e evita vazamento de identidades entre diferentes fases do experimento.

Essa organização possibilita a realização de experimentos de reconhecimento facial de forma reprodutível e controlada, simulando um cenário de cadastramento e validação de usuários em sistemas de controle de acesso. A base de dados sintética construída nesta etapa constitui o insumo fundamental para as etapas subsequentes da metodologia, incluindo a extração de características faciais, a geração de \textit{embeddings} e a avaliação da separabilidade entre identidades no espaço métrico \cite{ge2018hierarchical}.

%---------------------------------------------------------

\section{Modelagem do cadastro de identidades}

Após a construção e organização da base de dados sintética, procedeu-se à modelagem do cadastro de identidades a ser utilizado nos experimentos de reconhecimento facial. Essa modelagem teve como objetivo estruturar as informações de cada indivíduo de forma padronizada, facilitando o armazenamento, o processamento e a reprodutibilidade das etapas subsequentes da metodologia.

Cada identidade foi representada por um registro estruturado em formato \acs{JSON}, contendo os seguintes atributos: um identificador único (id), um nome fictício, o nome do arquivo da imagem facial associada e um campo destinado ao armazenamento da representação vetorial da face (\textit{embeddings}), inicialmente não preenchido. Essa estrutura permite a separação clara entre os dados descritivos da identidade e as informações extraídas automaticamente a partir das imagens faciais.

Com o intuito de organizar adequadamente os experimentos e evitar sobreposição de identidades entre diferentes fases de avaliação, o cadastro foi dividido em dois arquivos \acs{JSON} distintos. O primeiro arquivo corresponde ao conjunto de desenvolvimento, contendo as identidades destinadas às etapas de treino, validação e teste. Nesse arquivo, cada registro inclui um atributo adicional que indica a qual subconjunto a identidade pertence (treino, validação ou teste), respeitando a divisão previamente definida de 240, 30 e 30 identidades, respectivamente.

O segundo arquivo \acs{JSON} foi destinado exclusivamente ao conjunto de teste final, composto por 100 identidades distintas e não presentes no conjunto de desenvolvimento. Esse conjunto tem como finalidade permitir uma avaliação independente da abordagem proposta, simulando um cenário em que o sistema é exposto a identidades não utilizadas nas fases anteriores do processo experimental.

A adoção de arquivos \acs{JSON} distintos para o conjunto de desenvolvimento e para o teste final contribui para a organização do fluxo experimental, reduzindo riscos de vazamento de dados entre conjuntos e garantindo maior clareza na definição dos protocolos de avaliação. Além disso, o uso de um formato estruturado e amplamente suportado facilita a integração com os scripts de processamento, extração de \textit{embeddings} e análise de similaridade empregados nas etapas subsequentes da metodologia.

%---------------------------------------------------------

\section{Extração inicial de características faciais (baseline)}

Como etapa inicial do estudo de viabilidade, foi implementada uma abordagem baseada na extração de características geométricas do rosto, utilizando o modelo MediaPipe FaceMesh \cite{mlkit_face_mesh}, com o objetivo de validar o \textit{pipeline} de processamento e estabelecer uma linha de base (\textit{baseline}) para comparação com abordagens mais avançadas de reconhecimento facial. Essa etapa não tem como finalidade propor uma solução final, mas sim verificar a adequação do uso de representações vetoriais e métricas de similaridade no contexto do problema investigado.

Inicialmente, para cada imagem presente no conjunto de desenvolvimento, realizou-se a detecção automática da face por meio de um detector facial. Em seguida, a região facial detectada foi processada pelo modelo MediaPipe FaceMesh, o qual fornece um conjunto denso de pontos característicos (\textit{landmarks}) distribuídos ao longo da face, cada um associado a coordenadas espaciais tridimensionais \cite{mlkit_face_mesh}.

As coordenadas (x, y, z) correspondentes a cada \textit{landmark} foram então concatenadas, formando um vetor numérico de dimensão fixa que representa a geometria facial associada à imagem analisada. Esse vetor passou a ser utilizado como uma representação vetorial inicial da face, sendo armazenado no campo de \textit{embedding} do cadastro de identidades em formato \acs{JSON}.

A comparação entre diferentes faces foi realizada por meio da aplicação de uma métrica de similaridade entre os vetores gerados, permitindo o cálculo de distâncias entre pares de imagens \cite{ge2018hierarchical}. Essa abordagem possibilita observar o comportamento das distâncias para identidades iguais e distintas, servindo como referência experimental para a análise posterior de métodos baseados em aprendizado profundo e \textit{embedding} treinados especificamente para reconhecimento facial.

%---------------------------------------------------------

\section{Geração de variações sintéticas das imagens}

Com o objetivo de aumentar a diversidade das amostras disponíveis para cada identidade e simular variações comuns no processo de captura de imagens faciais em ambientes reais, foi realizada a geração de variações sintéticas das imagens originais presentes no conjunto de desenvolvimento. Essa etapa visa tornar o processo de avaliação mais robusto frente a alterações de condições que podem ocorrer em cenários de controle de acesso, como mudanças de iluminação, pequenas variações de pose e degradações na qualidade da imagem.

Para cada imagem facial original associada a uma identidade, foram aplicadas transformações artificiais controladas, resultando na criação de múltiplas amostras derivadas da mesma identidade. As transformações empregadas incluem operações geométricas, como rotações leves e ajustes de escala, bem como transformações fotométricas, como variações de brilho, contraste e saturação. Adicionalmente, foram aplicadas técnicas de degradação da imagem, incluindo a inserção de ruído e a aplicação de filtros de desfoque, com o objetivo de simular condições adversas de captura. A ~\ref{fig:thisperson} ilustra um exemplo de imagem facial original e suas respectivas variações sintéticas, evidenciando o efeito das transformações artificiais aplicadas para uma mesma identidade.


\begin{figure}[!htb]
\centering
\caption{\label{fig:thisperson} Variações sintéticas de uma identidade facial.}

\begin{tabular}{@{}cc@{}}
\begin{subfigure}[t]{0.32\textwidth}
\centering
\includegraphics[width=\linewidth]{Imagens/face1.jpg}
\caption{Imagem original}
\end{subfigure}
&
\begin{subfigure}[t]{0.32\textwidth}
\centering
\includegraphics[width=\linewidth]{Imagens/face2.jpg}
\caption{Variação 1}
\end{subfigure}
\\[2mm]
\begin{subfigure}[t]{0.32\textwidth}
\centering
\includegraphics[width=\linewidth]{Imagens/face3.jpg}
\caption{Variação 2}
\end{subfigure}
&
\begin{subfigure}[t]{0.32\textwidth}
\centering
\includegraphics[width=\linewidth]{Imagens/face4.jpg}
\caption{Variação 3}
\end{subfigure}
\end{tabular}

\caption*{\footnotesize Fonte: Elaborado pelo autor a partir de imagens sintéticas do projeto \textit{This Person Does Not Exist}, 2026}
\end{figure}


As imagens geradas por meio desse processo foram armazenadas em um diretório específico, mantendo-se a associação com o identificador da identidade original. Para refletir essa ampliação do conjunto de dados, foi criado um novo arquivo \acs{JSON} contendo múltiplos registros por identidade, cada um referenciando uma variação distinta da imagem facial, preservando-se o mesmo identificador único (id). Dessa forma, todas as variações de uma mesma identidade são tratadas como amostras positivas associadas a um único indivíduo \cite{ge2018hierarchical}.

A geração dessas variações sintéticas é particularmente relevante no contexto de abordagens baseadas em aprendizado de representações em espaço métrico, nas quais se busca reduzir a distância entre diferentes amostras da mesma identidade e aumentar a separação em relação a amostras de identidades distintas \cite{ge2018hierarchical}. Embora nesta etapa não seja realizado treinamento de redes neurais, o conjunto de imagens gerado fornece subsídios adequados para avaliar o comportamento de \textit{embeddings} faciais frente a variações de uma mesma identidade, conceito central em estratégias inspiradas em funções de perda do tipo \textit{triplet loss} \cite{kaya2019deep}.

%---------------------------------------------------------

\section{Extração de embeddings faciais por modelo pré-treinado}

Após a geração das variações sintéticas das imagens faciais, procedeu-se à extração de \textit{embeddings} faciais por meio de um modelo de reconhecimento facial pré-treinado baseado em aprendizado profundo. Essa etapa tem como objetivo representar cada face em um espaço vetorial no qual identidades iguais apresentem maior proximidade entre si, enquanto identidades distintas sejam mapeadas para regiões mais distantes, característica essencial para sistemas de reconhecimento facial baseados em métricas de similaridade \cite{ge2018hierarchical, abdullah2021face}.

Para essa finalidade, foi adotado um modelo amplamente utilizado na literatura e em aplicações práticas de reconhecimento facial, treinado previamente em grandes bases de dados de faces \cite{schroff2015facenet}. O modelo empregado segue a filosofia de aprendizado de representações em espaço métrico, sendo treinado com funções de perda do tipo \textit{margin-based}, conceitualmente relacionadas à \textit{triplet loss}, cujo objetivo é maximizar a separação entre diferentes identidades e reduzir a distância entre amostras da mesma identidade \cite{ge2018hierarchical}.

Cada imagem presente no conjunto de dados ampliado foi processada individualmente pelo modelo, sendo inicialmente realizada a detecção da face e, em seguida, a extração do vetor de características correspondente. O \textit{embedding} resultante consiste em um vetor numérico de dimensão fixa, capaz de capturar características discriminativas da face, de forma robusta a variações de iluminação, expressão facial e pequenas alterações de pose \cite{kaya2019deep}.

Os \textit{embeddings} extraídos foram associados às respectivas identidades e armazenados no cadastro estruturado em formato \acs{JSON}, substituindo o campo previamente reservado para a representação vetorial. Esse procedimento permitiu a construção de uma base de dados composta por múltiplas representações vetoriais por identidade, viabilizando a comparação entre amostras da mesma identidade e de identidades distintas nas etapas subsequentes da metodologia.

%---------------------------------------------------------

\section{Definição da métrica de similaridade}

Com os \textit{embeddings} faciais extraídos e associados às identidades, definiu-se o método de comparação entre amostras no espaço vetorial, conforme o paradigma de reconhecimento baseado em aprendizado métrico \cite{schroff2015facenet}.

Neste trabalho, a comparação entre dois \textit{embeddings} foi realizada por meio da distância cosseno, aplicada a vetores normalizados, produzindo um escore $d_{\cos}(\mathbf{x}, \mathbf{y})$ para cada par avaliado.

A decisão de correspondência foi determinada por um limiar $\tau$ (\textit{threshold}): comparações com $d_{\cos}(\mathbf{x}, \mathbf{y}) \leq \tau$ foram tratadas como correspondência (\textit{match}), enquanto valores superiores foram tratados como não correspondência. O valor de $\tau$ (Tau) foi considerado um parâmetro experimental e sua adequação foi avaliada no protocolo experimental por meio das métricas quantitativas apresentadas nas seções subsequentes.


%---------------------------------------------------------

\section{Protocolo experimental de avaliação}

O protocolo experimental adotado neste trabalho tem como finalidade avaliar o comportamento das representações faciais no espaço de \textit{embeddings} e verificar a capacidade da abordagem proposta em distinguir identidades distintas por meio de métricas de similaridade. Para isso, foram definidas estratégias de comparação entre amostras faciais, bem como métricas quantitativas para análise do desempenho do sistema, sem a realização de ajustes ou treinamentos adicionais dos modelos empregados.

As comparações foram organizadas a partir da definição de dois tipos de pares. Os pares genuínos correspondem a comparações entre amostras pertencentes à mesma identidade, incluindo imagens originais e variações sintéticas associadas a um mesmo identificador. Já os pares impostores correspondem a comparações entre amostras de identidades distintas. Essa distinção permite analisar o comportamento das distâncias de similaridade tanto em situações de correspondência legítima quanto em tentativas de correspondência indevida.

Para cada par de amostras, foi calculada a distância entre os \textit{embeddings} faciais utilizando a métrica definida na etapa anterior. A partir dessas distâncias, avaliou-se o impacto de diferentes valores do limiar de decisão (\textit{threshold}) na classificação das comparações como correspondências positivas ou negativas. Esse procedimento possibilita observar como variações no limiar influenciam o comportamento do sistema frente a pares genuínos e impostores \cite{jain2004biometric}.

Como métricas de avaliação, foram consideradas a taxa de falsa aceitação (\textit{False Acceptance Rate} - \acs{FAR}) e a taxa de falsa rejeição (\textit{False Rejection Rate} - \acs{FRR}). Essas métricas são amplamente utilizadas em sistemas biométricos e permitem caracterizar o compromisso entre segurança e usabilidade, aspecto fundamental em aplicações de controle de acesso \cite{jain2004biometric}. A análise conjunta dessas métricas fornece subsídios para a avaliação da separabilidade das identidades no espaço de \textit{embeddings} e para a definição de critérios de viabilidade da abordagem estudada.

Esse protocolo experimental foi aplicado tanto à abordagem inicial baseada em características geométricas quanto à abordagem baseada em \textit{embeddings} extraídos por modelo pré-treinado, permitindo uma avaliação consistente do comportamento das diferentes representações faciais consideradas neste estudo.

A metodologia apresentada definiu o conjunto de procedimentos adotados para a construção da base de dados, a implementação do pipeline de reconhecimento facial e o protocolo experimental de avaliação. A partir dessa configuração experimental, o capítulo a seguir apresenta os resultados obtidos e discute o comportamento do sistema proposto diante das métricas e cenários avaliados.
