\chapter{Conclusão}
\label{chap:conclusao}
Este trabalho avaliou a viabilidade do uso de reconhecimento facial baseado em \textit{embeddings} como alternativa aos cartões \acs{NFC} no controle de acesso e na cobrança de refeições no \acs{CEFET-MG}, com foco na validação conceitual e experimental da abordagem. Para isso, foi implementado um \textit{pipeline} de reconhecimento facial e analisado seu comportamento por meio de métricas biométricas e diferentes valores de \textit{threshold}.

Os resultados indicaram que os \textit{embeddings} extraídos por modelos pré-treinados organizam o espaço vetorial de forma consistente, apresentando boa separação entre identidades distintas e proximidade entre amostras da mesma identidade. A análise das métricas \acs{FAR} (\textit{False Acceptance Rate}) e \acs{FRR} (\textit{False Rejection Rate}) mostrou que é possível definir um limiar de decisão que estabelece um compromisso adequado entre segurança e usabilidade, conforme esperado em sistemas biométricos. Além disso, os experimentos no cenário de identificação 1:N demonstraram que o sistema é capaz de recuperar corretamente a identidade mais provável a partir da base de referências, reforçando a viabilidade técnica da abordagem.

Entretanto, o trabalho apresenta limitações importantes. A principal delas é o uso exclusivo de imagens faciais sintéticas, adotado por razões éticas e de conformidade com a \acs{LGPD}, o que não representa integralmente as condições de um ambiente real, sujeito a variações de iluminação, pose e qualidade de captura. Além disso, o estudo não contempla a integração com dispositivos físicos, nem a implementação de um sistema completo em tempo real, concentrando-se na validação do método de reconhecimento em si.

Nesse contexto, o \acs{PFC} II tem como objetivo dar continuidade a esta pesquisa por meio da implementação de um protótipo funcional de sistema de controle de acesso baseado em reconhecimento facial. Espera-se integrar o \textit{pipeline} desenvolvido a um ambiente mais próximo do real, incluindo captura por câmera, comunicação segura entre cliente e servidor, gerenciamento de identidades e avaliação do desempenho em condições práticas de uso.

Em síntese, o \acs{PFC} I cumpriu seu papel ao demonstrar, de forma experimental, que a abordagem de reconhecimento facial baseada em \textit{embeddings} é tecnicamente viável como alternativa aos cartões \acs{NFC}. O \acs{PFC} II representa o próximo passo, voltado à transformação dessa viabilidade em um sistema funcional e avaliável em um cenário aplicado.
