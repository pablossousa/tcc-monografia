\chapter{Introdução}

Sistemas de controle de acesso baseados em cartões de proximidade, como aqueles que utilizam tecnologias \acs{RFID} e \acs{NFC}, são amplamente empregados em ambientes institucionais por sua simplicidade operacional e facilidade de integração. No contexto do \acs{CEFET-MG}, esse tipo de solução é utilizado em diferentes serviços do cotidiano acadêmico, como o acesso às catracas da portaria, o empréstimo de livros na biblioteca e o pagamento de refeições no restaurante universitário. Embora funcionais, esses sistemas baseiam-se essencialmente na posse de um objeto físico, o que implica limitações práticas e de segurança, como perdas, extravios, compartilhamento indevido e a necessidade de reposição constante de cartões.

Como alternativa, sistemas biométricos buscam associar o processo de autenticação diretamente à identidade do indivíduo, explorando características físicas ou comportamentais, como a face ou a impressão digital \cite{jain2004biometric, bolle2013guide}. Entre as diferentes modalidades, o reconhecimento facial destaca-se por permitir autenticação sem contato físico e por utilizar dispositivos de captura amplamente disponíveis, como câmeras. Em abordagens modernas, a identidade não é verificada por comparação direta de imagens, mas por meio de representações vetoriais compactas, conhecidas como \textit{embeddings}, que permitem medir a similaridade entre faces no espaço vetorial \cite{schroff2015facenet, abdullah2021face}. Essa forma de representação está inserida no paradigma do aprendizado métrico, no qual o objetivo é organizar o espaço de características de modo que amostras da mesma identidade fiquem próximas entre si e amostras de identidades diferentes fiquem mais distantes \cite{bellet2015survey, ge2018hierarchical}.

O uso de técnicas de aprendizado profundo para reconhecimento facial, incluindo arquiteturas baseadas em redes siamesas e funções de perda do tipo \textit{Triplet Loss}, foi popularizado tanto na literatura quanto em materiais didáticos amplamente difundidos, como os apresentados por Andrew Ng \cite{ng2021deeplearning}. Esses métodos permitiram avanços significativos na qualidade dos \textit{embeddings} faciais e na robustez de sistemas de verificação e identificação. Ao mesmo tempo, o emprego de biometria levanta questões importantes relacionadas à privacidade e à proteção de dados pessoais, uma vez que características biométricas estão diretamente ligadas à identidade dos indivíduos \cite{north2020biometric}. No contexto brasileiro, tais preocupações são formalizadas pela Lei Geral de Proteção de Dados Pessoais (\acs{LGPD}), que classifica dados biométricos como dados pessoais sensíveis e impõe requisitos específicos para seu tratamento \cite{lgpd2018lei}.

Diversos autores já investigaram o uso de biometria em ambientes educacionais e em sistemas de controle de acesso, empregando diferentes modalidades e arquiteturas. Entretanto, essas propostas geralmente partem de premissas e contextos distintos, como o uso de sensores específicos ou o desenvolvimento de sistemas completos do zero. O presente trabalho diferencia-se por direcionar o foco à realidade do \acs{CEFET-MG} e por investigar a viabilidade de substituir os cartões físicos atualmente utilizados por um mecanismo de autenticação baseado em reconhecimento facial, concebido como um módulo adicional a sistemas já existentes.

Dessa forma, a pergunta de pesquisa que orienta este trabalho pode ser formulada da seguinte maneira: \textit{é possível substituir os cartões físicos utilizados na cobrança de refeições e no controle de acesso no âmbito do \acs{CEFET-MG} por um sistema baseado em reconhecimento facial, mantendo níveis adequados de confiabilidade e viabilidade técnica?} Para responder a essa questão, o estudo adota uma abordagem experimental fundamentada no uso de \textit{embeddings} faciais e métricas de similaridade, avaliando o comportamento do sistema sob diferentes critérios de decisão.

\section{Objetivo Geral}

O objetivo geral deste trabalho é avaliar a viabilidade tecnológica do uso de reconhecimento facial baseado em \textit{embeddings} como alternativa aos cartões \acs{NFC} atualmente utilizados no controle de acesso e na cobrança de refeições no \acs{CEFET-MG}.

\section{Objetivos Específicos}

Como desdobramento do objetivo geral, definem-se os seguintes objetivos específicos:
\begin{itemize}
    \item Realizar uma prospecção de técnicas e modelos de reconhecimento facial baseados em \textit{embeddings} e aprendizado métrico, considerando abordagens consolidadas na literatura.
    \item Construir uma base de dados experimental a partir de imagens faciais sintéticas geradas pelo projeto \textit{This Person Does Not Exist} \cite{thispersondoesnotexist}, de modo a evitar o uso de dados biométricos reais.
    \item Implementar um pipeline de reconhecimento facial utilizando modelos pré-treinados para extração de \textit{embeddings} faciais e métricas de similaridade no espaço vetorial.
    \item Avaliar experimentalmente a capacidade do sistema em verificar e identificar identidades, analisando métricas como \acs{FAR} e \acs{FRR} em função do limiar de decisão.
    \item Verificar se os resultados obtidos indicam a viabilidade técnica da substituição dos cartões físicos por um mecanismo de autenticação baseado em reconhecimento facial no contexto proposto.
\end{itemize}

\section{Organização do Trabalho}

Este trabalho está organizado da seguinte forma:
\begin{itemize}
    \item O \ref{chap:fundamentacao} apresenta a fundamentação teórica, abordando conceitos de RFID e \acs{NFC}, biometria, representação vetorial, aprendizado métrico, reconhecimento facial por \textit{embeddings}, métricas de avaliação biométrica e aspectos de privacidade e segurança.
    \item O \ref{chap:trabalhos-relacionados} discute os trabalhos relacionados, situando a proposta deste estudo no contexto da literatura.
    \item O \ref{chap:metodologia} descreve a metodologia adotada, incluindo a construção da base de dados sintética, o pipeline de reconhecimento e o protocolo experimental.
    \item O \ref{chap:resultados} apresenta os resultados obtidos e a respectiva discussão.
    \item O \ref{chap:conclusao} apresenta as conclusões e considerações finais do trabalho.
\end{itemize}
