\chapter{Resultados}
\label{chap:resultados}

Este capítulo apresenta os resultados obtidos a partir dos experimentos realizados com a base de dados sintética e os modelos de reconhecimento facial descritos na metodologia. 

Conforme detalhado no capítulo anterior, o \textit{pipeline} computacional foi inicialmente submetido a uma Prova de Conceito (\acs{PoC}) utilizando representações geométricas extraídas pelo modelo MediaPipe FaceMesh. Essa etapa cumpriu seu propósito ao confirmar o funcionamento integrado dos módulos de detecção facial, geração de vetores de características e cálculo de similaridade matemática em buscas 1:N. Uma vez validado o fluxo de trabalho, os testes quantitativos rigorosos de desempenho biométrico, que compõem o escopo central deste capítulo, foram conduzidos exclusivamente com os \textit{embeddings} gerados pelo modelo pré-treinado baseado em aprendizado profundo (InsightFace). 

Os resultados apresentados a seguir concentram-se na análise da separabilidade dessas representações faciais no espaço vetorial, na avaliação do desempenho biométrico por meio das métricas de falsa aceitação (\acs{FAR}) e falsa rejeição (\acs{FRR}), e na validação do processo de identificação baseado em similaridade, considerando diferentes valores de limiar de decisão. O objetivo é verificar, de forma objetiva, a viabilidade do uso do reconhecimento facial como alternativa ao controle de acesso baseado em cartões \acs{NFC} no contexto acadêmico.

%---------------------------------------------------------

\section{Separabilidade no espaço de embeddings}

Esta seção analisa a separabilidade das representações faciais no espaço de \textit{embeddings} a partir da distribuição das distâncias cosseno calculadas entre pares de imagens. O objetivo é verificar se amostras da mesma identidade apresentam maior similaridade entre si do que amostras pertencentes a identidades distintas.

Para isso, todas as combinações possíveis de pares foram comparadas e classificadas em dois grupos: pares da mesma identidade, formados por imagens diferentes associadas ao mesmo identificador, e pares de identidades diferentes, formados por imagens pertencentes a indivíduos distintos. No experimento realizado, foram obtidos 1800 pares da mesma identidade e 717.600 pares de identidades diferentes. Essa diferença numérica é esperada, uma vez que o número de pares impostores cresce de forma combinatória quando se consideram todas as combinações entre identidades distintas, enquanto os pares genuínos dependem apenas do número de variações disponíveis por identidade.

\begin{figure}[H]
    \centering
    \begin{minipage}{1.0\textwidth}
        \caption{\label{fig:result1} Distribuição das distâncias cosseno no espaço de \textit{embeddings}}
        \includegraphics[width=\textwidth]{Imagens/result1.png}
        \caption*{\footnotesize Fonte: Elaborado pelo autor, 2026.}
    \end{minipage}
\end{figure}

A \ref{fig:result1} apresenta a distribuição das distâncias cosseno para ambos os grupos. Observa-se que os pares da mesma identidade concentram-se em valores menores de distância, enquanto os pares de identidades diferentes apresentam distâncias predominantemente maiores, indicando separação entre os grupos no espaço de \textit{embeddings}. O limiar de referência de 0,3 é exibido apenas como apoio visual, ilustrando a região em que ocorreria a decisão de correspondência, sem ser utilizado nesta etapa para classificação.

De forma geral, os resultados indicam que o espaço de \textit{embeddings} apresenta separabilidade adequada entre identidades, constituindo a base para as análises quantitativas de desempenho apresentadas a seguir.

%---------------------------------------------------------

\section{Avaliação biométrica por FAR e FRR}

Nesta seção é avaliado o desempenho do sistema de reconhecimento facial por meio das métricas biométricas de falsa aceitação (\acs{FAR}) e falsa rejeição (\acs{FRR}), calculadas a partir das distâncias cosseno entre pares de \textit{embeddings} para diferentes valores de limiar de decisão (\textit{threshold}). O objetivo é analisar como a escolha do limiar influencia o comportamento do sistema em termos de segurança e confiabilidade.

Para essa avaliação, foram considerados 719.400 pares de comparação, obtidos a partir de 1200 amostras faciais com \textit{embeddings} válidos. Cada par foi classificado de acordo com sua identidade real (mesma identidade ou identidades distintas) e comparado com base na distância cosseno. A decisão de correspondência foi tomada comparando-se essa distância com um valor de \textit{threshold}, sendo considerada uma correspondência positiva quando a distância é inferior ao limiar.

O valor de 719.400 é consistente com a fórmula do total pares possíveis, uma vez que esse número corresponde a $\frac{1200 \times 1199}{2}$ $\Big(\frac{N\times (N-1)}{2}\Big)$ combinações possíveis entre vetores válidos.

\begin{table}[H]
\centering
\caption{Resultados de desempenho para diferentes valores de \textit{threshold}, incluindo TP, FP, TN, FN, FAR e FRR}
\label{tab:table1}
\begin{tabular}{c|cccc|cc}
\hline
\textbf{Threshold} & \textbf{TP} & \textbf{FP} & \textbf{TN} & \textbf{FN} & \textbf{FAR (\%)} & \textbf{FRR (\%)} \\
\hline
0,20 & 1800 & 0  & 717600 & 0 & 0,000 & 0,000 \\
0,30 & 1800 & 0  & 717600 & 0 & 0,000 & 0,000 \\
0,35 & 1800 & 0  & 717600 & 0 & 0,000 & 0,000 \\
0,40 & 1800 & 6  & 717594 & 0 & 0,001 & 0,000 \\
0,45 & 1800 & 38  & 717562 & 0 & 0,005 & 0,000 \\
0,50 & 1800 & 157 & 717443 & 0 & 0,022 & 0,000 \\
\hline
\end{tabular}
\caption*{Fonte: Elaborado pelo autor (2026).}
\end{table}

Os resultados são organizados em termos de quatro categorias: verdadeiros positivos (\acs{TP}), correspondentes a pares da mesma identidade corretamente aceitos; falsos negativos (\acs{FN}), pares da mesma identidade incorretamente rejeitados; falsos positivos (\acs{FP}), pares de identidades diferentes incorretamente aceitos; e verdadeiros negativos (\acs{TN}), pares de identidades diferentes corretamente rejeitados. A partir dessas quantidades, são calculadas as métricas \acs{FAR}, definida como a proporção de falsos positivos em relação ao total de pares impostores, e \acs{FRR}, definida como a proporção de falsos negativos em relação ao total de pares genuínos.

A \ref{tab:table1} apresenta os resultados obtidos para os diferentes valores de threshold avaliados. Verifica-se que, no intervalo de 0,20 a 0,35, o sistema não registrou ocorrências de \acs{FAR} ou \acs{FRR}. A partir de 0,40, observa-se o surgimento de falsas aceitações, evidenciado pelo aumento progressivo da \acs{FAR}, enquanto a \acs{FRR} permaneceu nula em todos os limiares testados.

Os resultados indicam elevada separabilidade entre as representações faciais, possibilitando a definição de um limiar que preserva a integridade das comparações genuínas e mantém controle rigoroso sobre aceitações indevidas.

%---------------------------------------------------------

\section{Definição do limiar de decisão (threshold)}

Com base na análise quantitativa realizada, definiu-se o limiar de decisão a ser utilizado no sistema. Considerando que, a partir de 0,40, passam a ocorrer falsas aceitações, optou-se por um valor dentro do intervalo que preserva erro zero para ambas as métricas.

Assim, foi adotado o threshold de 0,30 por representar o maior valor testado que mantém \acs{FAR} e \acs{FRR} iguais a zero. Essa escolha maximiza a margem de tolerância do sistema sem comprometer a segurança, sendo, portanto, utilizado como critério nas etapas subsequentes de validação e identificação.

%---------------------------------------------------------

\section{Análise qualitativa dos pares de comparação}

Esta seção apresenta uma análise qualitativa dos resultados obtidos, com o objetivo de complementar as métricas quantitativas discutidas anteriormente por meio da inspeção visual de pares de imagens faciais. Essa análise busca verificar se o comportamento observado nas distâncias entre \textit{embeddings} é consistente com a percepção visual das imagens comparadas, contribuindo para a validação empírica do sistema.

Foram selecionados exemplos representativos de pares da mesma identidade e de pares de identidades diferentes, considerando o limiar de decisão definido na seção anterior (distância cosseno inferior a 0,30 para correspondência positiva). Nos pares da mesma identidade, são comparadas imagens distintas associadas a um mesmo identificador, incluindo variações sintéticas decorrentes de transformações aplicadas às imagens originais. Nos pares de identidades diferentes, são comparadas imagens pertencentes a indivíduos distintos corretamente rejeitados pelo sistema.

\begin{figure}[H]
    \centering
    \begin{minipage}{1.0\textwidth}
        \caption{\label{fig:pessoas_iguais} Exemplo de par da mesma identidade corretamente aceitos}
        \includegraphics[width=\textwidth]{Imagens/pessoas_iguais.png}
        \caption*{\footnotesize Fonte: Elaborado pelo autor, 2026.}
    \end{minipage}
\end{figure}

\begin{figure}[H]
    \centering
    \begin{minipage}{1.0\textwidth}
        \caption{\label{fig:pessoas_diferentes} Exemplo de par de identidades diferentes corretamente rejeitados}
        \includegraphics[width=\textwidth]{Imagens/pessoas_diferentes.png}
        \caption*{\footnotesize Fonte: Elaborado pelo autor, 2026.}
    \end{minipage}
\end{figure}

As figuras apresentadas evidenciam que, nos casos classificados como pertencentes à mesma identidade, as imagens compartilham características faciais visuais compatíveis, mesmo diante de variações de iluminação, ruído ou pequenas alterações de aparência introduzidas artificialmente. Por outro lado, nos pares de identidades diferentes, observa-se divergência visual clara entre os rostos comparados, coerente com as maiores distâncias no espaço de \textit{embeddings} e com a decisão de não correspondência adotada pelo sistema.

De forma geral, a análise qualitativa confirma que as decisões baseadas em similaridade no espaço de \textit{embeddings} refletem adequadamente as diferenças e semelhanças perceptíveis entre as imagens faciais, reforçando a confiabilidade dos resultados quantitativos apresentados nas seções anteriores.

%---------------------------------------------------------

\section{Avaliação do processo de identificação (1:N)}

Nesta seção avalia-se o funcionamento do processo de identificação facial no cenário 1:N, no qual uma amostra de consulta é comparada com as representações faciais armazenadas na base de referência, com o objetivo de recuperar a identidade mais semelhante. A avaliação tem caráter funcional e ilustra a aplicação prática do critério de similaridade e do limiar definido nas seções anteriores ($\tau = 0{,}30$).

Para o teste, utilizou-se o subconjunto contendo 1200 amostras com \textit{embeddings} válidos. Uma amostra foi selecionada como consulta e seu \textit{embedding} foi comparado, por distância cosseno, com todas as demais amostras da base. Para evitar auto-comparação, removeu-se apenas a própria amostra de consulta do conjunto de referência, mantendo-se outras amostras da mesma identidade quando disponíveis. O resultado da identificação foi definido pela menor distância observada, e a correspondência foi aceita quando $d_{\cos}(x,y)\leq\tau$.

Como resultado, a busca recuperou uma amostra da mesma identidade como candidata mais próxima, com distância cosseno significativamente inferior a $\tau$, caracterizando uma correspondência genuína e coerente com o comportamento observado nas métricas \acs{FAR}/\acs{FRR}. Ressalta-se que este experimento não constitui uma avaliação estatística completa de desempenho 1:N (por exemplo, com múltiplas consultas e métricas de \textit{rank-1 accuracy}), mas serve como validação funcional do \textit{pipeline} de identificação em um cenário compatível com aplicações de controle de acesso.
