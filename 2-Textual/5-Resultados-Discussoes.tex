\chapter{Resultados}

Este capítulo apresenta os resultados obtidos a partir dos experimentos realizados com a base de dados sintética e os modelos de reconhecimento facial descritos na metodologia. Os resultados concentram-se na análise da separabilidade das representações faciais no espaço de \textit{embeddings}, na avaliação do desempenho biométrico por meio das métricas de falsa aceitação e falsa rejeição, e na validação do processo de identificação baseado em similaridade, considerando diferentes valores de limiar de decisão. O objetivo é verificar, de forma objetiva, a viabilidade do uso do reconhecimento facial como alternativa ao controle de acesso baseado em cartões \acs{NFC} no contexto acadêmico.

%---------------------------------------------------------

\section{Separabilidade no espaço de embeddings}

Esta seção analisa a separabilidade das representações faciais no espaço de embeddings a partir da distribuição das distâncias cosseno calculadas entre pares de imagens. O objetivo é verificar se amostras da mesma identidade apresentam maior similaridade entre si do que amostras pertencentes a identidades distintas.

Para isso, todas as combinações possíveis de pares foram comparadas e classificadas em dois grupos: pares da mesma identidade, formados por imagens diferentes associadas ao mesmo identificador, e pares de identidades diferentes, formados por imagens pertencentes a indivíduos distintos. No experimento realizado, foram obtidos 373 pares da mesma identidade e 37.028 pares de identidades diferentes. Essa diferença numérica é esperada, uma vez que o número de pares impostores cresce de forma combinatória quando se consideram todas as combinações entre identidades distintas, enquanto os pares genuínos dependem apenas do número de variações disponíveis por identidade.

\begin{figure}[H]
    \centering
    \begin{minipage}{1.0\textwidth}
        \caption{\label{fig:result1} Distribuição das distâncias cosseno no espaço de embeddings}
        \includegraphics[width=\textwidth]{Imagens/result1.png}
        \caption*{\footnotesize Fonte: Elaborado pelo autor, 2026.}
    \end{minipage}
\end{figure}

A \ref{fig:result1} apresenta a distribuição das distâncias cosseno para ambos os grupos. Observa-se que os pares da mesma identidade concentram-se em valores menores de distância, enquanto os pares de identidades diferentes apresentam distâncias predominantemente maiores, indicando separação entre os grupos no espaço de embeddings. O limiar de referência de 0,4 é exibido apenas como apoio visual, ilustrando a região em que ocorreria a decisão de correspondência, sem ser utilizado nesta etapa para classificação.

De forma geral, os resultados indicam que o espaço de embeddings apresenta separabilidade adequada entre identidades, constituindo a base para as análises quantitativas de desempenho apresentadas a seguir.

%---------------------------------------------------------

\section{Avaliação biométrica por FAR e FRR}

Nesta seção é avaliado o desempenho do sistema de reconhecimento facial por meio das métricas biométricas de falsa aceitação (FAR) e falsa rejeição (FRR), calculadas a partir das distâncias cosseno entre pares de embeddings para diferentes valores de limiar de decisão (threshold). O objetivo é analisar como a escolha do limiar influencia o comportamento do sistema em termos de segurança e confiabilidade.

Para essa avaliação, foram considerados 37.401 pares de comparação, obtidos a partir de 274 amostras faciais com embeddings válidos. Cada par foi classificado de acordo com sua identidade real (mesma identidade ou identidades distintas) e comparado com base na distância cosseno. A decisão de correspondência foi tomada comparando-se essa distância com um valor de threshold, sendo considerada uma correspondência positiva quando a distância é inferior ao limiar.

\begin{table}[H]
\centering
\caption{Resultados de desempenho para diferentes valores de \textit{threshold}, incluindo TP, FP, TN, FN, FAR e FRR}
\label{tab:table1}
\begin{tabular}{c|cccc|cc}
\hline
\textbf{Threshold} & \textbf{TP} & \textbf{FP} & \textbf{TN} & \textbf{FN} & \textbf{FAR (\%)} & \textbf{FRR (\%)} \\
\hline
0.20 & 373 & 0  & 37028 & 0 & 0.000 & 0.000 \\
0.30 & 373 & 0  & 37028 & 0 & 0.000 & 0.000 \\
0.35 & 373 & 0  & 37028 & 0 & 0.000 & 0.000 \\
0.40 & 373 & 0  & 37028 & 0 & 0.000 & 0.000 \\
0.45 & 373 & 2  & 37026 & 0 & 0.005 & 0.000 \\
0.50 & 373 & 32 & 36996 & 0 & 0.086 & 0.000 \\
\hline
\end{tabular}
\caption*{Fonte: Elaborado pelo autor (2026).}
\end{table}

Os resultados são organizados em termos de quatro categorias: verdadeiros positivos (TP), correspondentes a pares da mesma identidade corretamente aceitos; falsos negativos (FN), pares da mesma identidade incorretamente rejeitados; falsos positivos (FP), pares de identidades diferentes incorretamente aceitos; e verdadeiros negativos (TN), pares de identidades diferentes corretamente rejeitados. A partir dessas quantidades, são calculadas as métricas FAR, definida como a proporção de falsos positivos em relação ao total de pares impostores, e FRR, definida como a proporção de falsos negativos em relação ao total de pares genuínos.

A \ref{tab:table1} apresenta os resultados obtidos para diferentes valores de threshold. Observa-se que, para limiares entre 0,20 e 0,40, o sistema apresentou FAR e FRR iguais a zero, indicando ausência de erros tanto de aceitação indevida quanto de rejeição indevida nesse intervalo. A partir do limiar 0,45, passam a ocorrer falsas aceitações, refletidas no aumento gradual da FAR, enquanto a FRR permanece nula em todos os valores avaliados, indicando que nenhuma comparação genuína foi rejeitada.

Esses resultados evidenciam que o sistema apresenta alta separabilidade entre identidades, permitindo a definição de um limiar de decisão que elimina erros de falsa rejeição e mantém taxas de falsa aceitação extremamente baixas. Essa análise quantitativa fornece subsídios diretos para a escolha do threshold mais adequado, discutida na seção subsequente.

%---------------------------------------------------------

\section{Definição do limiar de decisão (threshold)}

%---------------------------------------------------------

\section{Análise qualitativa dos pares de comparação}

%---------------------------------------------------------

\section{Avaliação do processo de identificação (1 vs N)}