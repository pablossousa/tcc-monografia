\begin{otherlanguage}{english}
    \begin{abstract}
    \thispagestyle{empty}
This work investigates the feasibility of using face recognition based on \textit{embeddings} as an alternative to \acs{NFC} cards for access control and meal charging at CEFET-MG. The main objective is to validate, at a conceptual and experimental level, the application of metric learning techniques in an institutional scenario, considering performance and reliability aspects. For this purpose, a face recognition \textit{pipeline} was implemented using pre-trained models for feature extraction and a similarity metric in the vector space, with decisions based on a \textit{threshold}. The evaluation was carried out using a dataset composed of synthetic facial images, adopted for ethical reasons and to ensure compliance with the \acs{LGPD}, and included both verification experiments and identification experiments in the 1:N scenario. The system performance was analyzed through the biometric metrics \acs{FAR} (\textit{False Acceptance Rate}) and \acs{FRR} (\textit{False Rejection Rate}), as well as by observing the distribution of distances between genuine and impostor pairs. The results indicate that the \textit{embeddings} produce a representation space with good separation between different identities and consistency among samples of the same identity, allowing the definition of a decision threshold that establishes an adequate trade-off between security and usability. It is concluded that the proposed approach is technically feasible as an alternative to physical cards, although it is still restricted to an experimental environment, serving as a basis for the implementation of a functional prototype in PFC II.
        \\[0.5em]
        \textbf{Keywords:} face recognition; biometrics; \textit{embeddings}; access control; \acs{NFC}.
    \end{abstract}
\end{otherlanguage}
