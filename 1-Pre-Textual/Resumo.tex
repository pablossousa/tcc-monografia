\begin{abstract}
\thispagestyle{empty}
Este trabalho investiga a viabilidade do uso de reconhecimento facial baseado em \textit{embeddings} como alternativa aos cartões \acs{NFC} no controle de acesso e na cobrança de refeições no CEFET-MG. O objetivo principal é validar, em nível conceitual e experimental, a aplicação de técnicas de aprendizado métrico em um cenário institucional, considerando aspectos de desempenho e confiabilidade. Para isso, foi implementado um \textit{pipeline} de reconhecimento facial utilizando modelos pré-treinados para a extração de características e uma métrica de similaridade no espaço vetorial, com decisão baseada em um \textit{threshold}. A avaliação foi conduzida a partir de uma base de dados composta por imagens faciais sintéticas, adotada por razões éticas e de conformidade com a \acs{LGPD}, e incluiu tanto experimentos de verificação quanto de identificação no cenário 1:N. O desempenho do sistema foi analisado por meio das métricas biométricas \acs{FAR} (\textit{False Acceptance Rate}) e \acs{FRR} (\textit{False Rejection Rate}), além da observação da distribuição das distâncias entre pares genuínos e impostores. Os resultados obtidos indicam que os \textit{embeddings} produzem um espaço de representação com boa separação entre identidades distintas e consistência entre amostras da mesma identidade, permitindo a definição de um limiar de decisão que estabelece um compromisso adequado entre segurança e usabilidade. Conclui-se que a abordagem é tecnicamente viável como alternativa aos cartões físicos, embora ainda restrita a um ambiente experimental, servindo como base para a implementação de um protótipo funcional.
\\[0.5em]
\textbf{Palavras-chave:} reconhecimento facial; biometria; \textit{embeddings}; controle de acesso; \acs{NFC}.
\end{abstract}
